\documentclass[a4paper]{article}

\usepackage[shortlabels]{enumitem}
\usepackage{fancyhdr}
\usepackage[utf8]{inputenc}
\usepackage[MeX]{polski}
\usepackage{longtable} 
\usepackage{algpseudocode,algorithm,algorithmicx}
\usepackage{graphicx}
\usepackage{geometry}
\usepackage{lipsum}
\usepackage{url}
\usepackage{latexsym,amsmath,amssymb,amsthm}
\usepackage{hyperref}
\usepackage{abstract}


\geometry{top=2.5cm, bottom=2.5cm, left=2.5cm, right=2.5cm}
\title{Obliczenia naukowe\\Lista 1}
\author{Dominik Kaczmarek, nr albumu 261757}

\begin{document}`
\maketitle
\tableofcontents

\section{Rozpoznanie arytmetyki}
    \subsection{Opis problemu}
        Napisanie programu, który przy użyciu iteracji obliczy:
        \begin{enumerate}[a)]
        \item \(macheps\) (\emph{epsilon maszynowy}), czyli najmniejszą liczbę taką, że \(fl(1.0 + macheps) > 1.0\) 
        \item \(eta\) (\emph{liczba maszynowa}), czyli najmniejszą liczbę taką, że \(fl(eta) > 0.0\)
        \item \(MAX\), czyli największą liczbę dla danej precyzji 
        \end{enumerate}
        dla wszystkich dostępnych typów zmiennopozycyjnych \textbf{Float16}, \textbf{Float32}, \textbf{Float64} zgodnych ze standardem IEEE 754.
        Otrzymane wyniki należy póżniej porównać:
        \begin{enumerate}[a)]
        \item \(macheps\) z wartościami zwracanymi przez funkcje \(\bold{eps}(Float16)\), \(\bold{eps}(Float32)\), \(\bold{eps}(Float64)\), oraz z danymi zawartymi w pliku nagłówkowym \texttt{float.h} dowolnej instalacji języka C.
        \item \(eta\) z wartościami zwracanymi przez funkcje: \(\bold{nextfloat}(Float16(0.0))\), \(\bold{nextfloat}(Float32(0.0))\), \(\bold{nextfloat}(Float64(0.0))\), oraz z $MIN_{sub}$
        \item \(MAX\) z wartościami zwracanymi przez funkcje: \(\bold{floatmax}(Float16)\), \(\bold{floatmax}(Float32)\), \(\bold{floatmax}(Float64)\), oraz z danymi zawartymi w pliku nagłówkowym \texttt{float.h} dowolnej instalacji języka C lub z danymi z wykładu.
        \end{enumerate}

    \subsection{Rozwiązanie}
        W każdym algorytmie zaczynamy od przypisaniu zmiennej \(x\) wartości $1.0$ (\(Float16(1)\), \(Float32(1)\), \(Float64(1)\)). Następnie w pętli wykonujemy na zmiennej $x$ następujące operacje:
        \begin{enumerate}[a)]
        \item \(macheps\) | dzielimy $x/2$ dopóki spełniony jest warunek \(1 + x/2 > 1.0\). Po wyjściu z pętli \(x = macheps\).
        \item \(eta\) | dzielimy $x/2$ dopóki \(x/2 > 0.0\). Po wyjściu z pętli \(x = eta\).
        \item \(MAX\) | mnożymy \(2 \cdot x\) dopóki \(2 \cdot x < \infty\). Po wyjściu z pętli wiemy że \(2 \cdot x = \infty\), ale wciąż  $x \neq MAX$. Tworzymy nową zmienną pomocniczą $y=x$. Teraz dzielimy $y/2$ i pod $x$ przypisujemy \(x = x + y\) dopóki \(x < \infty\). Dopiero po zakończeniu tej pętli otzrymamy \(x = MAX\).
        \end{enumerate}
        
        Programy wyznaczające szukane wartości znajdują się w pliku \emph{z1.jl}.  
    \subsection{Wyniki}
        \begin{table}[!h]
        \centering
        \begin{tabular}{|l | l | l | l|}
        \hline
        typ & macheps & eps(typ) & float.h \\ \hline
        Float16 & $0.000977$ & $0.000977$ & --- \\
        Float32 & $1.1920929\cdot 10^{-7}$ & $1.1920929\cdot 10^{-7}$ & $1.192093\cdot 10^{-7}$ \\
        Float64 & $2.220446049250313\cdot 10^{-16}$ & $2.220446049250313\cdot 10^{-16}$ & $2.220446\cdot 10^{-16}$ \\
        \hline
        \end{tabular}
        \caption{\label{table:1}Zestawienie obliczonych $macheps$ z wynikami zwracanymi przez funkcję $eps$ oraz danymi z biblioteki \texttt{float.h}}
        \end{table}
        
        \begin{table}[!h]
        \centering
        \begin{tabular}{|l | l | l | l|}
        \hline
        typ & eta & nextfloat(typ(0.0)) & $MIN_{sub}$ \\ \hline
        Float16 & $6.0\cdot 10^{-8}$ & $6.0\cdot 10^{-8}$ & --- \\
        Float32 & $1.0\cdot 10^{-45}$ & $1.0\cdot 10^{-45}$ & $1.4\cdot 10^{-45}$ \\
        Float64 & $5.0\cdot 10^{-324}$ & $5.0\cdot 10^{-324}$ & $4.9\cdot 10^{-324}$ \\
        \hline
        \end{tabular}
        \caption{\label{table:2}Zestawienie obliczonego \emph{eta} z wbudowaną funckją \emph{nextfloat(typ(0.0))} oraz wartościami $MIN_{sub}$ (podane na wykładzie).}
        \end{table}
        
        \begin{table}[!h]
        \centering
        \begin{tabular}{|l | l | l | l|}
        \hline        
        typ & \emph{MAX} & floatmax(typ) & float.h \\ \hline
        Float16 & $6.55\cdot 10^{4}$ & $6.55\cdot 10^{4}$ & --- \\
        Float32 & $3.4028235\cdot 10^{38}$ & $3.4028235\cdot 10^{38}$ & $3.402823\cdot 10^{38}$ \\
        Float64 & $1.7976931348623157\cdot 10^{308}$ & $1.7976931348623157\cdot 10^{308}$ & $1.797693\cdot 10^{308}$ \\
        \hline
        \end{tabular}
        \caption{\label{table:3}Zestawienie obliczonego \emph{MAX} z wynikami zwracanymi przez funkcję $floatmax(typ)$ oraz danymi z biblioteki \texttt{float.h}.}
        \end{table}
        
        \begin{table}[!h]
        \centering
        \begin{tabular}{|l | l | l|}
        \hline        
        typ & floatmin(typ) & $MIN_{nor}$ \\ \hline
        Float32 & $1.1754944\cdot 10^{-38}$ & $1.2\cdot 10^{-38}$ \\
        Float64 & $2.2250738585072014\cdot 10^{-308}$ & $2.2\cdot 10^{-308}$ \\
        \hline
        \end{tabular}
        \caption{\label{table:3}Zestawienie obliczonego \emph{MAX} z wynikami zwracanymi przez funkcję $floatmax(typ)$ oraz danymi z biblioteki \texttt{float.h}.}
        \end{table}
            
    \subsection{Wnioski}
        \begin{itemize}
            \item Wyniki otrzymane z iteracyjnych algorytmów pokrywają się z wartościami zwracanymi przez funkcje wbudowane w język \emph{Julia}.
            \item Komputer nie jest w stanie przedstawić wszytskich liczb rzeczywistymi
            \item Im większa odległość od zera maszynowego tym mniej liczb znajduje się między pobliski liczbami całkowitymi
            \item Porównując \emph{eta} z $MIN_{sub}$ widzimy, że zgadza się tylko rząd wielkości
            \item \emph{floatmin(typ)} zwraca najmniejszą bezwzględną liczbę dla zadanego typu
            \item Porównując \emph{floatmin(typ)} z $MIN_{nor}$ widzimy, że zgadza się tylko rząd wielkość
            \item \emph{eta} i \emph{nextfloat(typ(0.0))} mają postać zdenormalizowaną, natomiast \emph{floatmin(typ)} zwraca wartość znormalizowaną
            \item \emph{macheps} = \(2 \cdot  \epsilon\)
            
        \end{itemize}

\section{Epsilon maszynowy Kahana}
    \subsection{Opis problemu}
        Obliczenie wyrażenia \(3 \cdot (4/3 - 1) - 1\) dla wszystkich dostępnych typów zmiennopozycyjnych \textbf{Float16}, \textbf{Float32}, \textbf{Float64} i sprawdzenie czy pokrywa się z epsilonem maszynowym odpowiednich arytmetyk.
        
    \subsection{Rozwiązanie}
    Używamy funkcji \(one(typ)\) do uzykania liczby 1 każdego z zadanych typów zmiennopozycyjnych.
    Następnie obliczamy wyrażenie \(3 \cdot (4/3 - 1) - 1\) dla każdego typu.
    
    Programy wyznaczające szukane wartości znajdują się w pliku \emph{z2.jl}.  
    
    \subsection{Wyniki i Interpretacja}
 
    \begin{table}[!h]
    \centering
    \begin{tabular}{|l | l | l|}
    \hline
    typ & macheps Kahana & eps(typ) \\ \hline
    Float16 & $-0.0009765625$ & $0.0009765625$ \\
    Float32 & $1.1920929\cdot 10^{-7}$ & $1.1920929\cdot 10^{-7}$ \\
    Float64 & $-2.220446049250313\cdot 10^{-16}$ & $2.220446049250313\cdot 10^{-16}$ \\
    \hline
    \end{tabular}
    \caption{\label{table:4}Zestawienie obliczonego \emph{macheps} Kahana dla różnych typów wraz z poprawnymi wartościami.}
    \end{table}
    
    Wartości bezwzględne wyników uzyskanych za pomocą obliczeń zgadzają się z poprawnymi epsilonami maszynowymi. Ujemne wyniki spowodowane są parzystością mantysy typów i faktem, że w tym wypadku w rozwinięciu dwójkowym liczby $4/{3}$ na ostatniej pozycji mantysy znajduje się $0$, a więc zgodnie z zasadą \textit{"round to even"} liczba zaokrąglana jest z niedomiarem, co przy 
    dalszych obliczeniach daje wynik ujemny.
    
    \subsection{Wnioski}
    Przez skończoną dokładność reprezentacji, niektóre równania dające w normalnej arytmetyce $0$, w naszym przypadku zwracają wyniki różne od 0.

\section{Rozmieszczenie liczb zmiennopozycyjnych}
    \subsection{Opis problemu}
    Sprawdzić eksperymentalnie w języku Julia, że w arytmetyce Float64 (arytmetyce double w standarcie IEEE 754) liczby zmiennopozycyjne są równomiernie rozmieszczone w $[1, 2)$ z krokiem \(\delta = 2^{-52}\).
    Sprawdzić jak rozmieszczone są liczby zmiennopozycyjne w przedziale $[\frac{1}{2} , 1)$, jak w przedziale $[2, 4)$ i jak mogą być przedstawione dla rozpatrywanego przedziału.
    
    \subsection{Rozwiązanie}
        \begin{enumerate}[a)]
        \item Przedział $[1, 2)$ \\
        Mając \(\delta = 2^{-52}\) dla każdego \(k = 1, 2,..., 2^{52}-1\) liczymy wartość \(x = 1 + k\delta\) i przedstawiamy $x$ w postaci bitowej używając funkcji \texttt{bitstring($x$)}.
        \item Przedział $[2, 4)$ \\
        Korzystjąc z intuicji powiększamy deltę dwukrotnie (bo przedział $[2, 4)$ znajduje się dalej od zera niż $[1, 2)$), czyli  \(\delta = 2^{-51}\). Robimy to samo co w punkcie \emph{a)} tylko dla \(\delta = 2^{-51}\).
        \item Przedział $[\frac{1}{2} , 1)$ \\
        Analogicznie, korzystjąc z intuicji tym razem pomniejszamy deltę dwukrotnie (bo przedział $[\frac{1}{2} , 1)$ znajduje się bliżej zera niż $[1, 2)$), czyli  \(\delta = 2^{-53}\). Robimy to samo co w punkcie \emph{a)} tylko dla \(\delta = 2^{-53}\).
        \end{enumerate}
    
    \subsection{Wyniki i interpretacja}

    \begin{longtable}{| c |}
    \hline
    $[\frac{1}{2}, 1]$\hspace{2cm} $\delta = 2^{-53}$ \\ \hline
    0011111111100000000000000000000000000000000000000000000000000001 \\
    0011111111100000000000000000000000000000000000000000000000000010 \\
    0011111111100000000000000000000000000000000000000000000000000011 \\
    0011111111100000000000000000000000000000000000000000000000000100 \\
    $\vdots$ \\
    0011111111101111111111111111111111111111111111111111111111111100 \\
    0011111111101111111111111111111111111111111111111111111111111101 \\
    0011111111101111111111111111111111111111111111111111111111111110 \\
    0011111111101111111111111111111111111111111111111111111111111111 \\
    \hline
    $[1, 2]$\hspace{2cm} $\delta = 2^{-52}$ \\ \hline
    0011111111110000000000000000000000000000000000000000000000000001 \\
    0011111111110000000000000000000000000000000000000000000000000010 \\
    0011111111110000000000000000000000000000000000000000000000000011 \\
    0011111111110000000000000000000000000000000000000000000000000100 \\
    $\vdots$ \\
    0011111111111111111111111111111111111111111111111111111111111100 \\
    0011111111111111111111111111111111111111111111111111111111111101 \\
    0011111111111111111111111111111111111111111111111111111111111110 \\
    0011111111111111111111111111111111111111111111111111111111111111 \\
    \hline
    $[2, 4]$\hspace{2cm} $\delta = 2^{-51}$ \\ \hline
    0100000000000000000000000000000000000000000000000000000000000001 \\
    0100000000000000000000000000000000000000000000000000000000000010 \\
    0100000000000000000000000000000000000000000000000000000000000011 \\
    0100000000000000000000000000000000000000000000000000000000000100 \\
    $\vdots$ \\
    0100000000001111111111111111111111111111111111111111111111111100 \\
    0100000000001111111111111111111111111111111111111111111111111101 \\
    0100000000001111111111111111111111111111111111111111111111111110 \\
    0100000000001111111111111111111111111111111111111111111111111111 \\
    \hline
    %\end{tabular}
    \caption{\label{table:5} Prezentacja bitowa wybranych liczb w danych przedziałach.}
    \end{longtable}
    
    Dzięki funkcji \texttt{bitstring($x$)} możemy zauważyć, że przy inkrementacji zmiennej $k$ w każdej z rubryk inkrementuje się wartość mantysy. Fakt, że sąsaidujące wiersze różnią się dokładnie o $1$ mówi nam, że intuicja z mnożeniem lub dzieleniem bazowej $\delta = 2^{-52}$ ($machepsu$) była poprawna. 
    
    \subsection{Wnioski}
    W arytmetyce \textbf{Float64} w każdym z przedziałów postaci $[2^{k}, 2^{k+1})$ dla \(k \in \mathbb{C}\), znajduje się dokładnie \(2^{52}\) liczb.
    Im większa odległość od \emph{zera maszynowego} tym mniej liczb znajduje się w przedziałach postaci $[k, k+1)$

\section{Nieodwracalność dzielenia}
    \subsection{Opis problemu}
        Znaleźć eksperymentalnie w arytmetyce \texttt{Float64} dwie liczby: 
        \begin{enumerate}[a)]
            \item $x$ takiego, że \(x \in (1, 2) \wedge x\cdot (1/{x})\neq 1\)
            \item $x$ takiego, że \(x > 0 \wedge x\cdot (1/{x})\neq 1\)
        \end{enumerate}
        
    \subsection{Rozwiązanie}
        \begin{enumerate}[a)]
            \item Zaczynając od \(x = Float64(1)\) zwiększamy $x$ korzystjąc z funkcji \texttt{nextfloat(x)} do momentu kiedy \(x\cdot (1/{x})\neq 1\)
            \item Zaczynając od \(x = Float64(0)\) zwiększamy $x$ korzystjąc z funkcji \texttt{nextfloat(x)} do momentu kiedy \(x\cdot (1/{x})\neq 1\)
        \end{enumerate}
    
    \subsection{Wyniki}
        \begin{enumerate}[a)]
            \item Najmniejsza znaleziona liczba to x = $1.000000057228997$ \\
            \(x \cdot \frac{1}{x} = 0.9999999999999999 \)
            \item Najmniejsza znaleziona liczba to x = $5.0 \cdot 10^{-324}$ \\
            \(x \cdot \frac{1}{x} = \infty \)
        \end{enumerate}
    
    \subsection{Wnioski}
    Zadanie pokazuje, że działania arytmetyczne na liczbach zmiennopozycyjnych mogą generować błędy związane z zaokrąglaniem wyliczonych wartości. Nawet z pozoru proste rachunki mogą dać nieprawidłowe wyniki. Użytkownik powinien maksymalnie upraszczać swoje algorytmy w celu zminimalizowania szansy wystąpienia błędu. 

\section{Iloczyn skalarny}
    \subsection{Opis problemu}
    Obliczenie iloczynu skalarnego danych wektorów $x$ i $y$ z wykorzystaniem czterech różnych algorytmów sumowania dla typów \texttt{Float32} i \texttt{Float64}. \\ \\
    $x = [2.718281828, -3.141592654, 1.414213562, 0.5772156649, 0.3010299957]$ \\
    $y = [1486.2497, 878366.9879, -22.37492, 4773714.647, 0.000185049]$ \\
    
    Algorytmy do sprawdzenia:
    \begin{enumerate}[a)]
    \item "w przód": $\sum_{i=1}^{n} x_iy_i$
    \item "w tył": $\sum_{i=n}^{1} x_iy_i$
    \item dodanie dodatnich liczb w porządku od największej do najmniejszej oraz ujemnych w porządku od najmniejszej do największej, a następnie dodanie do siebie obliczonych sum częściowych
    \item metoda przeciwna do sposobu \emph{3}
    \end{enumerate}
    
    \subsection{Rozwiązanie}
        Programy zawierające powyższe algorytmy znajdują się w pliku \texttt{z5.jl}.
    
    \subsection{Wyniki}
    \begin{table}[!h]
    \centering
    \begin{tabular}{|l | l | l | l | l|}
    \hline
    typ & a & b & c & d \\ \hline
    Float32 & $-0.4999443$ & $-0.4543457$ & $-0.5$ & $-0.5$ \\
    Float64 & $1.0251881368296672\cdot 10^{-10}$ & $-1.5643308870494366\cdot 10^{-10}$ & $0.0$ & $0.0$ \\
    \hline
    \end{tabular}
    \caption{\label{table:6}Iloczyn skalarny danych wektorów, wyliczony każdym z podanych algorytmami}
    \end{table}
    \noindent Rzeczywisty iloczyn skalarny $x$ i $y$ wynosi $-1.00657107000000\cdot 10^{-11}$. Żaden z otrzymanych wyników nie pasuje do prawidłowej wartości. Najbliższy do oryginału wynik dało zastosowanie algorytmu \texttt{a)} z arytmetyką \texttt{Float64}.  
    
    \subsection{Wnioski}
    Eksperyment pokazał, że kolejność wykonywanych obliczeń drastycznie wpływa na otrzymywany wynik. Przy sumowaniu dwóch liczb z których jedna jest znacząco większa od drugiej, ta większa pochłania mniejszą przez wynik staje się mniej dokładny.

\section{Przybliżenie funkcji}
    \subsection{Opis problemu}
    Obliczenie w arytmetyce \texttt{Float64} wartości dwóch funkcji \[f(x)=\sqrt{x^2 + 1} - 1\] \[g(x)=\frac{x^2}{\sqrt{x^2+1}+1}\] dla kolejnych wartości $x=8^{-1}, 8^{-2}, 8^{-3}\dots$.
    
    \subsection{Rozwiązanie}
    Iteracja po kolejnych potęgach ósemki i liczenie wartości $f$ i $g$ dla każdej z nich.
    Programy znajdują się w pliku \texttt{z6.jl}.
    
    \subsection{Wyniki i Interpretacja}
   
    \begin{table}[!h]
    \centering
    \begin{tabular}{|l | l | l|}
    \hline
    $x$ & $f(x)$ & $g(x)$ \\ \hline
    $8^{-1}$ & $0.0077822185373186414$ & $0.0077822185373187065$ \\
    $8^{-2}$ & $0.00012206286282867573$ & $0.00012206286282875901$ \\
    $8^{-3}$ & $1.9073468138230965\cdot 10^{-6}$ & $1.907346813826566\cdot 10^{-6}$ \\
    $8^{-4}$ & $2.9802321943606103\cdot 10^{-8}$ & $2.9802321943606116\cdot 10^{-8}$ \\
    $\vdots$ & $\vdots$ & $\vdots$ \\
    $8^{-8}$ & $1.7763568394002505\cdot 10^{-15}$ & $1.7763568394002489\cdot 10^{-15}$ \\
    $8^{-9}$ & $0.0$ & $2.7755575615628914\cdot 10^{-17}$ \\
    $\vdots$ & $\vdots$ & $\vdots$ \\
    $8^{-176}$ & $0.0$ & $6.4758\cdot 10^{-319}$ \\
    $8^{-177}$ & $0.0$ & $1.012\cdot 10^{-320}$ \\
    $8^{-178}$ & $0.0$ & $1.6\cdot 10^{-322}$ \\
    $8^{-179}$ & $0.0$ & $0.0$ \\
    \hline
    \end{tabular}
    \caption{\label{table:7}Wartości funkcji $f$ i $g$ dla kolejnych argumentów.}
    \end{table}
    
     Z analizy matematycznej wiemy, że funkcje $f$ i $g$ są sobie równe i dla \(x \in {8^{-1}, 8^{-2},\dots,8^{-8}}\) ich wartości są rzeczywiście zbliżone. Jednak dla $x < 8^{-8}$ funkcja $f$ zaczęła zwracać wartość $0.0$, podczas gdy funkcja $g$ jeszcze dla $x = 8^{-178}$ pokazuje wartość różną od zera ($1.6\cdot 10^{-322}$). Różnice stanowi fakt, że w funkcji $f$ odejmujemy $1$ od pierwiastka \(\sqrt{x^2 + 1}\) przez co $f$ operuje na wartościach bardzo bliskich $0$, co z kolei powoduje utratę cyfr znaczących.. Funkcja $g$ nie generuje takiego błędu przez co jest dużo bardziej dokładna niż funkcja $f$.
    \subsection{Wnioski}
    Obliczenia należy wykonywać w ten sposób, aby liczba cyfr znaczących przy kolejnych działaniach różniła się jak najmnniej. Pozwoli to uniknąć szybkiej kumulacji utarty dokładności.

\section{Przybliżenie pochodnej}
    \subsection{Opis problemu}
    Mając funkcję \[f(x)=\sin{x}+\cos{3x}\] obliczyć przybliżone wartości pochodnej $f$ w punkcie $x_0 = 1$ ze wzoru \[\tilde{f'}(x_0)=\frac{f(x_0 + h)-f(x_0)}{h}\] przy \(h \to 0\) i porównać je wynikami matematycznie wyznaczonej pochodnej. \[f'(x)=\cos({x})-3\cdot \sin({3x})\] \ 
    Obliczyć  błąd  $|f'(x_0)-\tilde{f'}(x_0)|$ dla $h=2^{-n}$ $(n=0,1,2,\dots,54)$.
    
    \subsection{Rozwiązanie}
    Program znajduje się w pliku \texttt{z7.jl}
    
    \subsection{Wyniki}
    \begin{table}[!h]
    \centering
    \begin{tabular}{|l | l | l | l|}
    \hline
    $h$ & $\tilde{f'}(1)$ & $|f'(1)-\tilde{f'}(1)|$ & $1+h$ \\ \hline
    $2^{0}$ & $2.0179892252685967$ & $1.9010469435800585$ & $2.0$ \\
    $2^{-1}$ & $1.8704413979316472$ & $1.753499116243109$ & $1.5$\\
    %$2^{-2}$ & $1.1077870952342974$ & $0.9908448135457593$ & $1.25$ \\
    $\vdots$ & $\vdots$ & $\vdots$ & $\vdots$  \\
    $2^{-16}$ & $0.11700383928837255$ & $6.155759983439424\cdot 10^{-5}$ & $1.0000152587890625$ \\
    $2^{-17}$ & $0.11697306045971345$ & $3.077877117529937\cdot 10^{-5}$ & $1.0000076293945312$ \\
    $\vdots$ & $\vdots$ & $\vdots$ & $\vdots$ \\
    $2^{-27}$ & $0.11694231629371643$ & $3.460517827846843\cdot 10^{-8}$ & $1.0000000074505806$ \\
    $2^{-28}$ & $0.11694228649139404$ & $4.802855890773117\cdot 10^{-9}$ & $1.0000000037252903$ \\
    $2^{-29}$ & $0.11694222688674927$ & $5.480178888461751\cdot 10^{-8}$ & $1.0000000018626451$\\
    $\vdots$ & $\vdots$ & $\vdots$ & $\vdots$ \\
    $2^{-36}$ & $0.116943359375$ & $1.0776864618478044\cdot 10^{-6}$ & $1.000000000014552$ \\
    $2^{-37}$ & $0.1169281005859375$ & $1.4181102600652196\cdot 10^{-5}$ & $1.000000000007276$ \\
    $\vdots$ & $\vdots$ & $\vdots$ & $\vdots$ \\
    $2^{-52}$ & $-0.5$ & $0.6169422816885382$ & $1.0000000000000002$ \\
    $2^{-53}$ & $0.0$ & $0.11694228168853815$ & $1.0$\\
    $2^{-54}$ & $0.0$ & $0.11694228168853815$ & $1.0$\\
    \hline
    \end{tabular}
    \caption{\label{table:8}Wartości funkcji $f$ i $g$ dla kolejnych argumentów.}
    \end{table}
    
        Wyniki obliczeń dla poszczególnych wartości $h$ przedstawiono poniżej (Tabela \ref{table:8}). Początkowo zmniejszanie wartości $h$ przynosi oczekiwane skutki i błędy w liczeniu przybliżonej pochodnej są mniejsze, najdokładniejszy wynik uzyskano dla $h = 2^{-28}$. Dalsze zmniejszanie $h$ nie poprawiło jednak dokładności obliczeń, wręcz przeciwnie, błędy zaczęły z powrotem rosnąć.
    
    \subsection{Wnioski}
    Przy prowadzeniu obliczeń należy wystrzegać się liczb bardzo bliskich zeru.
    \end{document}
