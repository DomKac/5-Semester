\documentclass[a4paper]{article}

\usepackage[shortlabels]{enumitem}
\usepackage{fancyhdr}
\usepackage[utf8]{inputenc}
\usepackage[MeX]{polski}
\usepackage{longtable} 
\usepackage{algpseudocode,algorithm,algorithmicx}
\usepackage{graphicx}
\usepackage{geometry}
\usepackage{lipsum}
\usepackage{url}
\usepackage{latexsym,amsmath,amssymb,amsthm}
\usepackage{hyperref}
\usepackage{abstract}
\usepackage{amsmath}
\usepackage{changepage}
\usepackage{float}
\usepackage{amsmath}
\usepackage{mathtools}
\newcommand\tab[1][1cm]{\hspace*{#1}}

\geometry{top=2.5cm, bottom=2.5cm, left=2.5cm, right=2.5cm}
\title{Obliczenia naukowe\\Lista 3}
\author{Dominik Kaczmarek, nr albumu 261757}

\begin{document}
\maketitle
\tableofcontents
\newpage

\section{Zadanie 1}
    \subsection{Opis problemu}
    Napisać funkcję rozwiązującą równanie $f(x) = 0$ metodą bisekcji. \\\\
    \texttt{function mbisekcji(f, a::Float64, b::Float64, delta::Float64, epsilon::Float64)}
    \\\\
    \textbf{Dane:}\\
    \texttt{f} – funkcja f(x) zadana jako anonimowa funkcja (ang. anonymous function), \\
    \texttt{a,b} – końce przedziału początkowego, \\
    \texttt{delta,epsilon} – dokładności obliczeń, \\\\
    \textbf{Wyniki:}\\
    \texttt{(r,v,it,err)} – czwórka, gdzie \\
    \tab \texttt{r} – przybliżenie pierwiastka równania $f(x) = 0$, \\
    \tab \texttt{v} – wartość $f(r)$, \\
    \tab \texttt{it} – liczba wykonanych iteracji, \\
    \tab \texttt{err} – sygnalizacja błędu\\
    \tab \tab 0 - brak błędu\\
    \tab \tab 1 - funkcja nie zmienia znaku w przedziale [a,b]\\
    
    \subsection{Rozwiązanie}
    Aby metoda bisekcji działała poprawnie, przedział początkowy \texttt{[a,b]} musi spełniać warunki:
    \begin{itemize}
        \item w \texttt{[a,b]} znajduje się dokładnie jedno miejsce zerowe $\equiv f(a) \cdot f(b) < 0$
        \item funkcja \texttt{f} jest ciągła w \texttt{[a,b]}
    \end{itemize}
    \textbf{Lista kroków :}
    \begin{enumerate}[I]
        \item Obliczamy długość przedziału \texttt{e = b - a} oraz środek aktualnego przedziału \texttt{r = (a + e/2)}
        \item Jeżeli \texttt{r} różni się od rzeczywistego miejsca zerowego co najwyżej o \texttt{delta}  $ \equiv $ \texttt{e} $<$ \texttt{delta} \\
              lub wartość \texttt{f(r)} rózni się od \texttt{0} najwyżej o \texttt{epislon} $\equiv$ \texttt{f(r)} $ < $ \texttt{epsilon}, \\
              to kończymy obliczenia i zwracamy wynik \texttt{(r,v,it,0)}, w przeciwnym wypadku przechodzimy do punktu \texttt{(III)}
        \item Patrzymy, czy \texttt{f(r) i f(a)} mają rózne znaki $\equiv$ \texttt{f(r) * f(a) < 0}:
            \begin{itemize}
                \item Jeżeli \textbf{tak} to w przedziale \texttt{[a,r]} znajduje się miejsce zerowe, dlatego zmniejszamy obszar poszukiwań podstawiając \texttt{b = r}.
                \item Jeżeli \textbf{nie} to miesjce zerowe znajduje się w przedziale \texttt{[r,b]}, dlatego zmniejszamy obszar poszukiwań podstawiając \texttt{a = r}.
            \end{itemize}
            Przechodzimy do punktu \texttt{(I)}\\
    \end{enumerate}
    Dokładny algorytm znajduje się w pliku \texttt{functions.jl}.  

\section{Zadanie 2}
    \subsection{Opis problemu}
    Napisać funkcję rozwiązującą równanie $f(x) = 0$ metodą Newtona. \\
    \texttt{function mstycznych(f,pf,x0::Float64, delta::Float64, epsilon::Float64, maxit::Int)}
    \\\\
    \textbf{Dane:}\\
    \texttt{f, pf} – funkcją $f(x)$ oraz pochodną $f'(x)$ zadane jako anonimowe funkcje, \\
    \texttt{x0} – przybliżenie początkowe, \\
    \texttt{delta,epsilon} – dokładności obliczeń, \\
    \texttt{maxit} – maksymalna dopuszczalna liczba iteracji, \\\\
    \textbf{Wyniki:}\\
    \texttt{(r,v,it,err)} – czwórka, gdzie \\
    \tab \texttt{r} – przybliżenie pierwiastka równania $f(x) = 0$, \\
    \tab \texttt{v} – wartość $f(r)$, \\
    \tab \texttt{it} – liczba wykonanych iteracji, \\
    \tab \texttt{err} – sygnalizacja błędu\\
    \tab \tab 0 - metoda zbieżna \\
    \tab \tab 1 - nie osiągnięto wymaganej dokładności w \texttt{maxit} iteracji, \\
    \tab \tab 2 - pochodna bliska zeru \\
    
    \subsection{Rozwiązanie}

\section{Zadanie 3}
    \subsection{Opis problemu}
    Napisać funkcję rozwiązującą równanie $f(x) = 0$ metodą siecznych. \\
    \texttt{function msiecznych(f, x0::Float64, x1::Float64, delta::Float64, epsilon::Float64, maxit::Int)}
    \\\\
    \textbf{Dane:}\\
    \texttt{f} – funkcja f(x) zadana jako anonimowa funkcja (ang. anonymous function), \\
    \texttt{x0,x1} – przybliżenia początkowe, \\
    \texttt{delta,epsilon} – dokładności obliczeń, \\
    \texttt{maxit} – maksymalna dopuszczalna liczba iteracji, \\\\
    \textbf{Wyniki:}\\
    \texttt{(r,v,it,err)} – czwórka, gdzie \\
    \tab \texttt{r} – przybliżenie pierwiastka równania $f(x) = 0$, \\
    \tab \texttt{v} – wartość $f(r)$, \\
    \tab \texttt{it} – liczba wykonanych iteracji, \\
    \tab \texttt{err} – sygnalizacja błędu\\
    \tab \tab 0 - metoda zbieżna \\
    \tab \tab 1 - nie osiągnięto wymaganej dokładności w \texttt{maxit} iteracji, \\
    
    \subsection{Rozwiązanie}
    
\section{Zadanie 4}
    \subsection{Opis problemu}
    \subsection{Rozwiązanie}
    \subsection{Wyniki i interpretacja}
    \subsection{Wnioski}
\section{Zadanie 5}
    \subsection{Opis problemu}
    \subsection{Rozwiązanie}
    \subsection{Wyniki i interpretacja}
    \subsection{Wnioski}
\section{Zadanie 6}
    \subsection{Opis problemu}
    \subsection{Rozwiązanie}
    \subsection{Wyniki i interpretacja}
    \subsection{Wnioski}
\end{document}


\neq 
\approx \appr\to \leftarrow
